% First controller for smart.city.kempten LEGO(R) Model
% INNER CITY FRONT (BASE PLATE 1)
\documentclass[border=10pt]{standalone}
\usepackage[european,siunitx]{circuitikz}
\begin{document}

\begin{circuitikz}
 \ctikzset{multipoles/thickness=2}
 \ctikzset{multipoles/external pins thickness=2}

\ctikzset{multipoles/dipchip/width=3}
 \draw (5,6) node[dipchip,
 num pins=8,
 hide numbers,
 no topmark,
 external pins width=0.0,
 external pad fraction=4](display){Parking + Waste};
 \draw [color=red] (display.pin 1) -- ++(-1,0) node[vcc, font=\small]{USB Power};
 \draw [color=black] (display.pin 2) -- ++(-1,0) node[ground, font=\small]{};
 \node [left, font=\tiny] at (display.pin 8) {SDA};
 \node [left, font=\tiny] at (display.pin 7) {SCL};

\newcommand\multiplexer[4]{
 \ctikzset{multipoles/dipchip/width=2}
 \draw (#3,#4) node[dipchip,
 num pins=24,
 hide numbers,
 no topmark,
 external pins width=0.0,
 external pad fraction=4](#1){#2};

 \draw [color=red] (#1.pin 1) -- ++(-1.5,0) node[vcc, font=\small]{3V3};
 \draw [color=black] (#1.pin 2) -- ++(-1.5,0) node[ground, font=\small]{};
 \node [right, font=\tiny] at (#1.pin 1) {VIN};
 \node [right, font=\tiny] at (#1.pin 2) {GND};
 \node [right, font=\tiny] at (#1.pin 3) {SDA};
 \node [right, font=\tiny] at (#1.pin 4) {SCL};
 \node [right, font=\tiny] at (#1.pin 9) {SD0};
 \node [right, font=\tiny] at (#1.pin 10) {SC0};
 \node [right, font=\tiny] at (#1.pin 11) {SD1};
 \node [right, font=\tiny] at (#1.pin 12) {SC1};
 \node [left, font=\tiny] at (#1.pin 13) {SD2};
 \node [left, font=\tiny] at (#1.pin 14) {SC2};
 \node [left, font=\tiny] at (#1.pin 15) {SD3};
 \node [left, font=\tiny] at (#1.pin 16) {SC3};
 \node [left, font=\tiny] at (#1.pin 17) {SD4};
 \node [left, font=\tiny] at (#1.pin 18) {SC4};
 \node [left, font=\tiny] at (#1.pin 19) {SD5};
 \node [left, font=\tiny] at (#1.pin 20) {SC5};
 \node [left, font=\tiny] at (#1.pin 21) {SD6};
 \node [left, font=\tiny] at (#1.pin 22) {SC6};
 \node [left, font=\tiny] at (#1.pin 23) {SD7};
 \node [left, font=\tiny] at (#1.pin 24) {SC7};
}

\newcommand\sensor[3]{
 \ctikzset{multipoles/dipchip/width=0.8}
 \draw (#1,#2) node[dipchip,
 num pins=4,
 hide numbers,
 no topmark,
 external pins width=0.0,
 external pad fraction=4,
 rotate=90](#3){$#3$};
 \node [below, font=\tiny] at (#3.pin 4) {SDA};
 \node [below, font=\tiny] at (#3.pin 3) {SCL};
}


\multiplexer{M1}{MUX 0x70}{3}{0}

\foreach \x in {6,...,1} {
 \sensor{8.5-\x*1.5}{-5}{P_\x}
 }

\draw [color=orange] (P_6.pin 4) |- (M1.pin 9) -- (M1.pin 9){};
\draw [color=blue] (P_6.pin 3) |- (M1.pin 10) -- (M1.pin 10){};

\draw [color=orange] (P_5.pin 4) |- (M1.pin 11) -- (M1.pin 11){};
\draw [color=blue] (P_5.pin 3) |- (M1.pin 12) -- (M1.pin 12){};

\draw [color=orange] (P_4.pin 4) |- ++(0,0.8) -| ++(2.25,0) |- (M1.pin 13) -- (M1.pin 13){};
\draw [color=blue] (P_4.pin 3)  |- ++(0,0.7) -| ++(1.8,0)|- (M1.pin 14) -- (M1.pin 14){};

\draw [color=orange] (P_3.pin 4) |- ++(0,0.5) -| ++(1,0) |- (M1.pin 15) -- (M1.pin 15){};
\draw [color=blue] (P_3.pin 3)  |- ++(0,0.4) -| ++(0.55,0)|- (M1.pin 16) -- (M1.pin 16){};

\draw [color=orange] (P_2.pin 4) |- (M1.pin 17) -- (M1.pin 17){};
\draw [color=blue] (P_2.pin 3) |- (M1.pin 18) -- (M1.pin 18){};

\draw [color=orange] (P_1.pin 4) |- (M1.pin 19) -- (M1.pin 19){};
\draw [color=blue] (P_1.pin 3) |- (M1.pin 20) -- (M1.pin 20){};


\multiplexer{M2}{MUX 0x71}{11}{-11}

\foreach \x in {8,...,7} {
 \sensor{17.5-\x*1.5}{-11}{P_\x}
 }

\draw [color=orange] (P_7.pin 4) |- ++(0,3.9) -| ++(6.2,0) |- (M2.pin 23) -- (M2.pin 23){};
\draw [color=blue] (P_7.pin 3)  |- ++(0,3.7) -| ++(5.4,0)|- (M2.pin 24) -- (M2.pin 24){};

\draw [color=orange] (P_8.pin 4) |- ++(0,0.2) -| ++(-0.5,0) |- (M2.pin 11) -- (M2.pin 11){};
\draw [color=blue] (P_8.pin 3)  |- ++(0,0.4) -| ++(-1.25,0)|- (M2.pin 12) -- (M2.pin 12){};


\multiplexer{M3}{MUX 0x72}{0.2}{-11}

\foreach \x in {1,...,3} {
 \sensor{-3+\x*1.5}{-16}{W_\x}
 \draw (-3.25+\x*1.5,-18) to[empty led] (-2.75+\x*1.5,-18);
 \draw [color=red] (-3.25+\x*1.5,-18) -- ++(-0.25,0) |- (-3.5+\x*1.5,-19);
 \draw (-2.75+\x*1.5,-18) -- ++(0.25,0) |- (-2.5+\x*1.5,-19.25);
 }

\draw [color=red] (-5,-19) node[vcc, font=\small]{USB Power} to[R=1.5\si{\kilo\ohm}, font=\small] ++(3,0) -- ++(3,0); 
\draw (-1, -19.25) -- ++(3.0,0) node[ground, font=\small]{};


\draw [color=orange] (W_1.pin 4) |- ++(0,0.4) -| ++(-0.55,0) |- (M3.pin 9) -- (M3.pin 9){};
\draw [color=blue] (W_1.pin 3)  |- ++(0,0.5) -| ++(-1,0)|- (M3.pin 10) -- (M3.pin 10){};

\draw [color=orange] (W_2.pin 4) |- ++(0,0.7) -| ++(-1.6,0) |- (M3.pin 11) -- (M3.pin 10){};
\draw [color=blue] (W_2.pin 3)  |- ++(0,0.8) -| ++(-2.05,0)|- (M3.pin 12) -- (M3.pin 12){};

\draw [color=orange] (W_3.pin 4) |- ++(0,0.5) -| ++(0.9,0) |- (M3.pin 13) -- (M3.pin 13){};
\draw [color=blue] (W_3.pin 3)  |- ++(0,0.4) -| ++(0.45,0)|- (M3.pin 14) -- (M3.pin 14){};

\ctikzset{multipoles/dipchip/width=3}

\draw (12,1) node[dipchip,
 num pins=40,
 hide numbers,
 no topmark,
 external pins width=0.0,
 external pad fraction=4 ](P){PICO 2 W};

\draw (P.pin 38) -- ++(1.5,0) node[ground, font=\small]{};
\draw [color=red] (P.pin 40) -- ++(1.5,0) node[vcc, font=\small]{USB Power};
\draw [color=red] (P.pin 36) -- ++(0.5,0) node[vcc, font=\small]{3V3};
\node [right, font=\tiny] at (P.pin 1) {I2C0 SDA};
\node [right, font=\tiny] at (P.pin 2) {I2C0 SCL};
\node [right, font=\tiny] at (P.pin 4) {I2C1 SDA};
\node [right, font=\tiny] at (P.pin 5) {I2C1 SCL};
\node [left, font=\tiny] at (P.pin 40) {VBUS};
\node [left, font=\tiny] at (P.pin 39) {VSYS};
\node [left, font=\tiny] at (P.pin 38) {GND};
\node [left, font=\tiny] at (P.pin 36) {3V3};

\draw [color=orange] (display.pin 8) -- ++(1.6,0) |- (P.pin 1) -- (P.pin 1){};
\draw [color=blue] (display.pin 7) -- ++(1.4,0) |- (P.pin 2) -- (P.pin 2){};

\draw [color=orange] (M1.pin 3) -- ++(-0.5,0) |- (P.pin 4) -- (P.pin 4){};
\draw [color=blue] (M1.pin 4) -- ++(-0.7,0) |- (P.pin 5) -- (P.pin 5){};

\draw [color=orange] (M2.pin 3) -- ++(-0.5,0) |- (P.pin 4) -- (P.pin 4){};
\draw [color=blue] (M2.pin 4) -- ++(-0.7,0) |- (P.pin 5) -- (P.pin 5){};

\draw [color=orange] (M3.pin 3) -- ++(-0.5,0) |- (P.pin 4) -- (P.pin 4){};
\draw [color=blue] (M3.pin 4) -- ++(-0.7,0) |- (P.pin 5) -- (P.pin 5){};

\end{circuitikz}

\end{document}
